\chapter*{Методические замечания}
\addcontentsline{toc}{chapter}{Методические замечания}

При составлении лабораторной работы следует придерживаться
определённой структуры. Здесь рассматриваются лабораторные работы,
выполняемые на компьютере.

\section{Основные положения}
\label{sec:method:general}

\begin{itemize}
\item Рабочий каталог лабораторной работы должен иметь унифицированную
  структуру (см. \ref{sec:method:work:lab}).
\item Исходный код программ, отчёты по лабораторным работам и
  т.д. размещаются в системе контроля версий git (см. Лабораторную
  работу №~\ref{ch:git}).
\item Выполнение лабораторной работы документируется в виде скринкаста.
\item При проверке лабораторных работ следует использовать понятные
  критерии оценки.
\end{itemize}

\section{Рабочее пространство для лабораторной работы}
\label{sec:method:work:lab}

При выполнении лабораторной работы следует придерживаться структуры рабочего пространства.

\begin{itemize}
\item Рабочее пространство по предмету располагается в следующей иерархии:

\begin{minted}[]{bash}
~/work/
└── <учебный год>/
    └── <название предмета>/
        └── <код предмета>/
\end{minted}

Например, для 2021--2022 учебного года и предмета <<Операционные
системы>> (код предмета \texttt{os-intro}) структура каталогов примет следующий вид:

\begin{minted}[]{bash}
~/work/study
└── 2021-2022/
    └── Операционные системы/
        └── os-intro/
\end{minted}

\item Название проекта на хостинге git имеет вид:
\begin{verbatim}
study_<учебный год>_<код предмета>
\end{verbatim}
\item Например, для 2021--2022 учебного года и предмета «Операционные системы» (код предмета \texttt{os-intro}) название проекта примет следующий вид:
\begin{verbatim}
study_2021-2022_os-intro
\end{verbatim}


\item Каталог для лабораторных работ имеет вид \texttt{labs}.
\item Каталоги для лабораторных работ имеют вид \texttt{lab<номер>}, например: \texttt{lab01}, \texttt{lab02} и т.д.
\item Каталог для групповых проектов имеет вид \texttt{project-group}.
\item Каталог для персональных проектов имеет вид \texttt{project-personal}.
\item Если проектов несколько, то они нумеруются подобно лабораторным работам.
\item Этапы проекта обозначаются как \texttt{stage<номер>}.
\end{itemize}

\section{Шаблон для рабочего пространства}
\label{sec:method:template}

\begin{itemize}
\item Репозиторий: \url{https://github.com/yamadharma/course-directory-student-template}.
\end{itemize}

\subsection{Сознание репозитория курса на основе шаблона}
\label{sec:orgf3f8c19}

\begin{itemize}
\item Репозиторий на основе шаблона можно создать либо вручную, через web-интерфейс, либо с помощью утилит \texttt{gh} (см. \href{../notes/20210804144000-github_утилиты_команднои_строки.org}{github: утилиты командной строки}).
\item Создание с помощью утилит.
\item Создание выглядит следующим образом:
\begin{minted}[]{shell}
gh repo create <new-repo-name> --template="<owner/template-repo>"
\end{minted}
\item Например, для 2021--2022 учебного года и предмета «Операционные системы» (код предмета \texttt{os-intro}) создание репозитория примет следующий вид:
\begin{minted}[]{shell}
mkdir -p ~/work/study/2021-2022/"Операционные системы"
cd ~/work/study/2021-2022/"Операционные системы"
gh repo create study_2021-2022_os-intro --template=yamadharma/course-directory-student-template --public
git clone --recursive git@github.com:<owner>/study_2021-2022_os-intro.git os-intro
\end{minted}
\end{itemize}

\subsection{Настройка каталога курса}
\label{sec:org8b36581}

\begin{itemize}
\item Перейдите в каталог курса:
\begin{minted}[]{shell}
cd ~/work/study/2021-2022/"Операционные системы"/os-intro
\end{minted}

\item Удалите лишние файлы:
\begin{minted}[]{shell}
rm package.json
\end{minted}

\item Создайте необходимые каталоги:
\begin{minted}[]{shell}
make COURSE=os-intro
\end{minted}

\item Отправьте файлы на сервер:
\begin{minted}[]{shell}
git add .
git commit -am 'feat(main): make course structure'
git push
\end{minted}
\end{itemize}


\printbibliography[heading=subbibliography,resetnumbers=true]

%%% Local Variables:
%%% mode: latex
%%% coding: utf-8-unix
%%% End:

