
\chapter*{Введение}
\label{cha:intro}
\iftexht
\else
    \addcontentsline{toc}{chapter}{Введение}
\fi    

% Связь компьютеров в сеть изменила парадигму обработки информации. На
% место единичных компьютеров пришли их объединения. Стал актуальным
% тезис: <<Компьютер есть сеть>>. Главное назначение сетей на
% сегодняшний момент --- передача информации. В настоящее время
% существует множество сетевых технологий. И хотя все они движутся к
% совместному синтезу, оценить современное состояние невозможно без
% понимания генезиса сетевых технологий.

Объединение компьютеров в сеть изменило парадигму обработки
информации. Главное назначение сетей на сегодняшний момент~---
передача информации. В настоящее время существует множество сетевых
технологий со своими особенностями, критериями применимости
и~пр. Оценить современное состояние сферы телекоммуникаций невозможно
без понимания генезиса, принципов функционирования существующих систем
и сетей связи.


Учебное пособие предназначено для студентов, обучающихся по программе
дополнительного образования «Информационно-телекоммуникационные
системы». В рамках инновационной образовательной программы,
реализованной в РУДН в 2008--2009 гг. на кафедре систем
телекоммуникаций, разработан одноимённый учебно"=методический комплекс
(УМК), в состав которого входит электронный учебник. Программа
дополнительного образования является авторской и включает в себя набор
последовательно взаимоувязанных специальных дисциплин.

На программе могут обучаться студенты, не имеющие специального
образования, например, обучающиеся по направлениям «Автоматизация и
управление», «Математика, прикладная математика», «Физика»,
«Менеджмент», «Экономика», «Экономика и управление на предприятии (по
отраслям производства)».  Курс является составляющей модуля программы
дополнительной профессиональной подготовки «Основы управления
инфокоммуникациями», которая включает также курсы: «Архитектура и
принципы построения современных сетей и систем телекоммуникаций»,
«Введение в управление инфокоммуникациями», «Корпоративные
информационные системы».



% \chapter*{Структура книги}
Учебное пособие логически разбито на две части. 
Первая часть изучает протоколы компьютерных сетей согласно структуре
эталонной модели ISO/OSI.   Основное внимание уделяется двум стекам
протоколов~--- \index*{Ethernet} и \index*{TCP/IP}. Вторая часть рассматривает
современное состояние сетей телекоммуникаций.

В первой главе даются и объясняются такие базовые понятия систем
телекоммуникаций, как протокол, интерфейс, служба. Даётся обзор
существующих сетей связи, сетевых сервисов. Рассматриваются структура
и основные аспекты деятельности стандартизирующих организаций.

Во второй главе освещаются общие принципы построения модели
взаимодействия открытых систем (ISO/OSI), иерархия протоколов
различных стеков протоколов по отношению к модели ISO/OSI.

В третьей главе рассматриваются методы и технологии физического уровня
модели ISO/OSI. В частности, даётся обзор возможных сред передачи (в
том числе и стандарты кабельной системы), методов кодирования сигнала.

В четвёртой главе изучаются методы и протоколы доступа к среде, а
также технологии сетей (\index*{Ethernet}, \index{Ethernet!Fast
  Ethernet}Fast Ethernet, \index{Ethernet!Gigabit Ethernet}Gigabit
Ethernet, Wireless Networks, \index*{WiMAX} и т.д.). Упор делается на
стандарты IEEE 802.x.

В пятой главе рассматриваются протоколы межсетевого уровня стека
протоколов \index*{TCP/IP}. Особое внимание уделяется протоколу IP:
приведены формат кадра IP, схемы и правила IP-адресации
(\index{IP!IPv4}IPv4 и \index{IP!IPv6}IPv6).  В этой же главе
отдельным пунктом представлена проблема маршрутизации: классификация
алгоритмов маршрутизации, протоколы статической (\index*{iproute2},
click) и динамической (\index*{RIP}, \index*{OSPF}, \index*{BGP})
маршрутизации, сфера их применения. Кратко рассматриваются другие
протоколы межсетевого уровня стека протоколов \index*{TCP/IP}, их
назначение.

Шестая глава посвящена протоколам транспортного уровня стека
протоколов \index*{TCP/IP}: TCP, \index*{UDP}, \index*{DCCP}, \index*{SCTP}.

В седьмой главе описываются протоколы верхних уровней стека
\index*{TCP/IP}, а именно два протокола сеансового уровня
\index*{DNS} и \index*{ENUM}, ответственные за адресацию.

% С расширением услуг, предоставляемых сетями передачи данных, и
% изменением характера передаваемых данных (уменьшение доли трафика
% чисто файловых протоколов, таких как FTP, в пользу протоколов передачи
% мультимедийных данных) критичным становится обеспечение качества
% обслуживания (QoS) передачи. 

В восьмой главе вводятся базовые понятия \index*{QoS}, рассматриваются
специальные решения обеспечения \index*{QoS}, такие как организация
виртуальных каналов в \index*{ATM}, а также решения для IP-сетей и
\index*{Ethernet} --- организация виртуальных каналов при помощи меток
(\index*{MPLS}), разбиение трафика на классы в соответствии с
приоритетами каждого типа трафика и определение политик обслуживания
этих классов трафика (\index*{DiffServ} и \index*{IntServ}). В связи с
этим рассматриваются инструменты классификации и маркировки пакетов, а
также механизмы планирования и выравнивания трафика.

Девятая глава посвящена мультисервисным сетям. Также в ней описываются
основные подходы к построению сетей следующего поколения
(\index*{NGN}). В исторической ретроспективе рассматриваются два
основных подхода к построению конвергентных сетей~---
\index*{Softswitch} и \index*{IMS}, даётся сравнительный обзор их
концепций, освещаются архитектурные особенности и основные протоколы
обоих подходов.


%%% Local Variables: 
%%% mode: latex
%%% TeX-master: "../default"
%%% coding: utf-8-unix
%%% End: 


% LocalWords:  ISO OSI Ethernet TCP IP стандартизирующих IEEE SS ALOHA CSMA VG
% LocalWords:  Fast Gigabit Token Ring AnyLAN Wireless Networks WiMAX Bluetooth
% LocalWords:  FDDI ISDN Frame Relay IPv iproute click RIP OSPF BGP IGRP ARP
% LocalWords:  RARP ICMP IGMP UDP псевдозаголовок DCCP SCTP DNS ENUM FTP QoS
% LocalWords:  ATM MPLS DiffServ IntServ CoS ToS DSCP EXP WFQ RED LLQ NGN IMS
% LocalWords:  мультисервисным Softswitch
