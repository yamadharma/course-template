\thispagestyle{empty}

% \begin{onehalfspace}


\begin{minipage}[t]{0.4\linewidth}
\begin{flushleft}
УДК  004.051 (075.8) %004.272 (076.5)
\\
ББК  018.2*32.973
\\
\ \qquad О 60
\\
\end{flushleft}
\end{minipage}
\hfill
\begin{minipage}[t]{0.4\linewidth}
\begin{flushright}
Утверждено \\
РИС Учёного совета \\
Российского университета \\
дружбы народов \\
\end{flushright}
\end{minipage}

\vspace{1cm}

\begin{flushleft}
Рецензенты: \\
доцент, кандидат физико-математических наук, зав. сектором
телекоммуникаций УИТОиСТС РУДН, Ловецкий К.\,П.,\\
доцент, кандидат физико-математических наук, с.н.с. ЛИТ ОИЯИ
Стрельцова О.\,И.
% доктор технических наук, профессор, заведующий кафедрой сетей связи и
% систем телекоммуникаций МТУСИ  Пшеничников А.\,П., \\
% кандидат технических наук, доцент кафедры сетей связи СПбГУТ им. проф.
% М.\,А. Бонч-Бруевича  Кучерявый Е.\,А., \\
% PhD, профессор департамента электроники и коммуникаций
% Технологического университета Тампере (Финляндия)  Кучерявый Е.\,А.
\end{flushleft}
\bigskip
\bigskip
\bigskip


\noindent
\begin{minipage}[t]{0.08\linewidth}
\begin{flushleft}
\mbox{}\\
\textbf{О-60} \\
\end{flushleft}
\end{minipage}
\hfill
\begin{minipage}[t]{0.9\linewidth}
\begin{flushleft}
  \textbf{Авторский коллектив: \bookauthor{}.} \\
  \textbf{\booktitle{}: лабораторные работы} : учебное пособие~/
  \bookauthor{}.~---
  Москва : РУДН, \bookyear{}.~--- \pageref{MyLastPage} с. : ил. \\
\end{flushleft}
\end{minipage}

%\bigskip
%ISBN 

\bigskip

% Рассматриваются принципы построения систем и сетей телекоммуникаций,
% основные технологии локальных сетей, принципы и средства межсетевого
% взаимодействия, принципы построения и функционирования глобальных
% сетей. Описано функционирование и основные характеристики коммутаторов
% и маршрутизаторов, приводятся примеры конфигурирования устройств их
% проверки и отладки.

% Данный учебник издание рекомендуется Государственным образовательным
% учреждением высшего профессионального образования «Московский
% технический университет связи и информатики» к использованию в
% образовательных учреждениях, реализующих образовательные программы
% высшего профессионального образования, по дисциплине «Сети и системы
% передачи информации» по направлению подготовки специалистов «090302
% Информационная безопасность телекоммуникационных систем».
Данное учебное пособие рекомендуется для проведения лабораторных работ
по курсу <<Операционные системы>> для направлений 02.03.02
«Фундаментальная информатика и информационные технологии», 02.03.01
«Математика и компьютерные науки», 38.03.05 «Бизнес-информатика»,
01.03.02 «Прикладная математика и информатика», 09.03.03 «Прикладная
информатика».

\vfill

\hfill
\begin{minipage}[t]{0.4\linewidth}
\begin{flushleft}
УДК  004.451(075.8) %004.272 (076.5)
\\
ББК  018.2*32.973
\end{flushleft}
\end{minipage}



\vfill

\noindent
\begin{minipage}[t]{0.4\linewidth}
  \begin{flushleft}
    ISBN  978-5-209-07626-1
  \end{flushleft}
\end{minipage}
\hfill
\begin{minipage}[t]{0.6\linewidth}
  \begin{flushleft}
    \copyright{} %
    \bookauthorrev{}, \bookyear{} \\
    \copyright{} %
    Российский университет дружбы народов, % Издательство,
    \bookyear{}
  \end{flushleft}
\end{minipage}

% \end{onehalfspace}

\clearpage

%%% Local Variables: 
%%% mode: latex
%%% TeX-master: "../default"
%%% coding: utf-8-unix
%%% End: 

% LocalWords:  empty УДК ББК РИСО ОИЯИ Багинян УИТО РУДН Ловецкий Кулябов ISBN
% LocalWords:  Королькова
