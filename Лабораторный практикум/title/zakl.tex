
\chapter*{Заключение}
\label{cha:end}
\iftexht
\else
    \addcontentsline{toc}{chapter}{Заключение}
\fi    

Достижения техники за последнее десятилетие привели к настоящему буму
в области телекоммуникаций. Связь, находившаяся в статическом состоянии
с середины 1980-х гг., сегодня превратилась в бурно
развивающуюся отрасль.  Сегодняшним клиентам рынка
инфокоммуникационных услуг требуется широкий класс различных служб и
приложений, предполагающий большое разнообразие протоколов, технологий
и скоростей передачи. В существующей ситуации на рынке
инфокоммуникационных услуг сети перегружены: они переполнены
многочисленными интерфейсами клиентов, сетевыми слоями и
контролируются слишком большим числом систем управления. При эволюции
к прозрачной сети главной задачей является упрощение сети~--- это
требование рынка и технологии.

На сегодняшний день развитие инфокоммуникационных услуг
осуществляется в основном в рамках компьютерной сети Интернет,
доступ к услугам которой происходит через традиционные сети связи. В
то же время в ряде случаев услуги Интернета, ввиду ограниченных
возможностей её транспортной инфраструктуры, не отвечают современным
требованиям, предъявляемым к услугам информационного общества. В связи
с этим развитие инфокоммуникационных услуг требует решения задач
эффективного управления информационными ресурсами с одновременным
расширением функциональности сетей связи. В свою очередь, это
стимулирует процесс интеграции Интернета и сетей связи.


%%% Local Variables: 
%%% mode: latex
%%% TeX-master: "../default"
%%% coding: utf-8-unix
%%% End: 


